\documentclass[assd_tpf_main.tex]{subfiles}

\begin{document}

\section{L\'ineas de investigaci\'on}

En primera instancia, se realiz\'o una investigaci\'on acerca 
del estado del arte de este problema que se busca resolver.
Este tema es extensivamente explicado en \textit{Image inpainting: Overview and recent advances} por Christine Guillemot y Olivier Le Meur \cite{b3}.

\subsection{Definici\'on del problema de inpainting}

En primer lugar se procede a definir matemáticamente el problema que en este trabajo se aspira a resolver. 

Una imagen $I$ puede definirse de la siguiente manera
\begin{equation}
I(\vec x): R^{n} \to R^{m}
\end{equation}
Donde $I$ es una función que mapea un pixel $\vec x=(x,y)$ a un color $R^{m}$. Como la imagen es bidimensional, $n=2$, y utilizando un esquema RGB se tiene $m=3$.

En el problema de inpainting se entiende que la imagen $I$ fue degrada por un operador $M$, lo cual genera una imagen nueva $F$ que tiene la imagen con algunos pixeles cambiados (que pertenecen a una región $U$ conocida). Se suele escribir entonces:
\begin{equation}
F=M(I)
\end{equation}

Lo que se busca en principio es reconstruir la imagen original $I$ a partir de $F$ conociendo $M$ (qu\'e p\'ixeles fueron removidos y se quieren reconstruir). Este problema en el sentido estricto no tiene solución, se suele decir entonces que no es un problema bien definido con una unica solución.
Entonces el objetivo es lograr una aproximación de $I$ (llamese $I'$) que logre ser lo más parecida posible a $I$.

Otra consecuencia de este mal condicionamiento del problema es c\'omo definir el \'exito a partir de $I'$ y de $F$. Dado que se desconoce la imagen original $I$, que ser\'ia el resultado ideal, no existe una m\'etrica cuantitativa que determine la calidad de una reconstrucci\'on dada, y por lo tanto se debe recurrir a evaluaciones subjetivas. En general, lo que se espera como resultado es una imagen que parezca f\'isicamente plausible, y que parezca natural al ojo humano.

\subsection{Primeras ideas}
Existen una gran cantidad de técnicas para afrontar el problema.


\subsection{Métodos basados en difusión}


\subfile{difusion.tex}


\subsection{Métodos basados en ejemplares}
Una segunda categor\'ia de m\'etodos para resolver el problema de inpainting se basa en 
asumir que la imagen $I$ posee ciertas probabilidades de estacionareidad y auto similaridad 
entre sus distintas regiones. Asumiendo que esto es cierto, se procede a rellenar la parte
de la imagen que falta a partir del uso de ejemplares, o parches, de la region que se tiene.

Estos m\'etodos, entonces, tienen a grandes rasgos dos criterios a desarrollar:
\begin{itemize}
	\item c\'omo elegir por qu\'e secci\'on del contorno se comienza a rellenar, o qu\'e
	regi\'on debe rellenarse a continuaci\'on
	\item c\'omo elegir el parche con el cual se rellenar\'a dicha regi\'on
\end{itemize}

Un enfoque por fuerza bruta de estos problemas puede resultar excesivamente costoso en tiempo, ya que implicar\'ia potencialmente comparar una regi\'on de la imagen con todas las dem\'as. Por lo tanto, existen diversos criterios que pueden tomarse en este sentido.

Uno de ellos es desarrollado por Barnes, Shechtman, Finkelstein y Goldman, quienes desarrollaron un algoritmo con el nombre de PatchMatch \cite{patchmatch}.  El mismo
hace uso de la asunci\'on de ciertas propiedades de coherencia en la imagen para reducir
el n\'umero de comparaciones que debe hacerse, sin perder la efectividad del algoritmo.
 
Una vez elegido un lugar donde se quiere rellenar, PatchMatch comienza comparando con algunas regiones aleatorias de la imagen original, y a partir del resultado m\'as satisfactorio que obtiene hace dos suposiciones:
\begin{itemize}
	\item si se tiene un buen match para una regi\'on que se quiere rellenar, es probable 
	que, si existe un parche que es un match a\'un mejor, se encuentre cerca del bueno (es
	decir, se asume que estamos en la zona correcta de la imagen, y s\'olo debemos
	terminar de acercarnos al match \'optimo)
	\item si un parche es un buen match para una cierta regi\'on a rellenar, es probable que
	tambi\'en sea un buen match para regiones cercanas
\end{itemize}

De esta manera, seg\'un los autores del algoritmo, se logra obtener muy buenos resultados con menos comparaciones. La limitaci\'on que tiene este algoritmo es que no puede dejarse
de introducir comparaciones random incluso habiendo encontrado un ``buen'' match,
puesto que se corre el riesgo de quedarse en un m\'inimo local, que no sea el m\'inimo
global.

\subsection{M\'etodos h\'ibridos}
\end{document}