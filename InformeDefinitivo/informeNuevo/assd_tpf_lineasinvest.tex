\documentclass[assd_tpf_main.tex]{subfiles}

\begin{document}

\section{L\'ineas de investigaci\'on}

En primera instancia, se realiz\'o una investigaci\'on acerca 
del estado del arte de este problema que se busca resolver.
Este tema es extensivamente explicado en \textit{Image inpainting: Overview and recent advances} por Christine Guillemot y Olivier Le Meur \cite{b3}.

\subsection{Definici\'on del problema de inpainting}

En primer lugar se procede a definir matemáticamente el problema que en este trabajo se aspira a resolver. 

Una imagen $I$ puede definirse de la siguiente manera
\begin{equation}
I(\vec x): R^{n} \to R^{m}
\end{equation}
Donde $I$ es una función que mapea un pixel $\vec x=(x,y)$ a un color $R^{m}$. Como la imagen es bidimensional, $n=2$, y utilizando un esquema RGB se tiene $m=3$.

En el problema de inpainting se entiende que la imagen $I$ fue degrada por un operador $M$, lo cual genera una imagen nueva $F$ que tiene la imagen con algunos pixeles cambiados (que pertenecen a una región $U$ conocida). Se suele escribir entonces:
\begin{equation}
F=M(I)
\end{equation}

Lo que se busca en principio es reconstruir la imagen original $I$ a partir de $F$ conociendo $M$ (qu\'e p\'ixeles fueron removidos y se quieren reconstruir). Este problema en el sentido estricto no tiene solución, se suele decir entonces que no es un problema bien definido con una unica solución.
Entonces el objetivo es lograr una aproximación de $I$ (llamese $I'$) que logre ser lo más parecida posible a $I$.

Otra consecuencia de este mal condicionamiento del problema es c\'omo definir el \'exito a partir de $I'$ y de $F$. Dado que se desconoce la imagen original $I$, que ser\'ia el resultado ideal, no existe una m\'etrica cuantitativa que determine la calidad de una reconstrucci\'on dada, y por lo tanto se debe recurrir a evaluaciones subjetivas. En general, lo que se espera como resultado es una imagen que parezca f\'isicamente plausible, y que parezca natural al ojo humano.

\subsection{Primeras ideas}
Existen una gran variedad de técnicas para afrontar el problema. Dentro de todas ellas se distinguen dos subgrupos principales, que son los m\'etodos basados en difusión, y los métodos basados en ejemplares. Como no existe ninguna ventaja determinante de una variante y la otra en este trabajo práctico se procedió a investigar ambas familias de métodos. Acorde al tipo de imagen puede ser más conveniente un metodo u otro, incluso en algunos casos se puede usar una mezcla de ambos.

\subsubsection{Metodos basados en difusión}
La primera categoria de métodos se denomina difusión, y la idea fundamental proviene de propagar información con restricciones usando ecuaciónes diferenciales parciales. La idea consiste en mediante una ecuación propagar sobre la región $U$ con pixeles cambiados (del exterior al interior) información que permita reconstruirla de una manera acorde.
En este método juega un papel fundamental qu\'e ecuación utilizar para realizar la propagación, en particular sobre qu\'e direcciones propagar más, y sobre cu\'ales menos. En general este m\'etodo suele dar resultados de mayor efectividad cuando la imagen tiende a tener patrones homog\'eneos sin cambios de textura muy abruptos.  Más adelante se describirá una variante del método con mayor detalle.

\subsubsection{Métodos basados en ejemplares}

Una segunda categor\'ia de m\'etodos para resolver el problema de inpainting se basa en 
asumir que la imagen $I$ posee ciertas probabilidades de estacionareidad y auto similaridad 
entre sus distintas regiones. Partiendo de este supuesto, se procede a rellenar la parte
de la imagen que falta a partir del uso de ejemplares, o parches, de la region que se tiene.

Estos m\'etodos, entonces, tienen a grandes rasgos dos criterios a desarrollar:
\begin{itemize}
	\item c\'omo elegir por qu\'e secci\'on del contorno se comienza a rellenar, o qu\'e
	regi\'on debe rellenarse a continuaci\'on
	\item c\'omo elegir el parche con el cual se rellenar\'a dicha regi\'on
\end{itemize}


\subsubsection{M\'etodos h\'ibridos}
De los dos m\'etodos vistos hasta ahora, basados en difusi\'on o en ejemplares, ninguno es claramente superior al otro para todos los casos, sino que en ciertas circunstancias puede funcionar mejor uno o el otro.

En l\'ineas generales, con los m\'etodos basados en difusi\'on se obtienen buenos resultados para huecos peque\~nos y esparsamente distribuidos. Lo mismo aplica para estructuras con bordes definidos. Por otro lado, los m\'etodos basados en ejemplares funcionan bien para regiones con texturas o patrones regulares.

Como es razonable esperar que una imagen tenga tanto estructuras como texturas, surge una familia de algoritmos que busca aprovechar las ventajas de ambos m\'etodos seg\'un corresponda. Se los conoce, por lo tanto, como m\'etodos h\'ibridos.

A grandes rasgos, existen dos formas de implementar esta idea:

\begin{itemize}
	\item identificar qu\'e partes de la imagen corresponden a estructuras y qu\'e 
	partes a texturas, rellenar por difusi\'on o por ejemplares seg\'un corresponda, 
	y combinar los resultados obtenidos
	\item combinar ambos m\'etodos mediante el uso de una \'unica funci\'on de
	energ\'ia (m\'as all\'a del alcance de este trabajo)
\end{itemize}


\end{document}