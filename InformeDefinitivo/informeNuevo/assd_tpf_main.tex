\documentclass[conference]{IEEEtran}
\IEEEoverridecommandlockouts

\usepackage{listings}
\usepackage{float}
\usepackage[spanish, es-tabla, es-nodecimaldot]{babel}
\usepackage{cite}
\usepackage{amsmath,amssymb,amsfonts}
\usepackage{algorithmic}
\usepackage{graphicx}
\usepackage{textcomp}
\usepackage{xcolor}
\usepackage{subfiles}
\lstset{basicstyle=\small\ttfamily,columns=fullflexible}

% preambulo:
%\usepackage[utf8]{inputenc}
% caracteres utf8 (tildes, enie) sin tener que usar comandos


%% NO AGREGAR PAQUETES ANTES DE ESTO, ES IMPORTANTE QUE BABEL ESTE PRIMERO

%%%%%%%%%%%%%%%%%%%%%%%%%%%%%%%%%
%% PAQUETES EXTRA %%%%%%%%%%%%%%%
%%%%%%%%%%%%%%%%%%%%%%%%%%%%%%%%%

\usepackage{subfiles}

\usepackage{cite}
\usepackage{amsmath,amssymb,amsfonts}
\usepackage{algorithmic}
\usepackage{textcomp}
\usepackage{xcolor}

\usepackage{steinmetz} % comando \phase{}
\usepackage{units} % permite usar nicefrac
\usepackage{graphicx} % importar imagenes
\usepackage{float} % posicion H para floats
\usepackage[colorinlistoftodos]{todonotes}


\usepackage[a4paper, total={6in, 8in}]{geometry} 
% margenes correctos en subarchivos

\setlength{\parindent}{10pt}			%cuanta sangria al principio de un parrafo
\usepackage{indentfirst}				%pone sangria al primer parrafo de una seccion

\def\BibTeX{{\rm B\kern-.05em{\sc i\kern-.025em b}\kern-.08em
    T\kern-.1667em\lower.7ex\hbox{E}\kern-.125emX}}


%%%%%%%%%%%%%%%%%%%%%%%%%%%%%%%%%%%%%%%%%%%%%%%%%%%%%%%%%%%
%% NO AGREGAR PAQUETES DESPUES DE ESTO, ES IMPORTANTE QUE HYPERREF ESTE ULTIMO
\usepackage[hidelinks]{hyperref} % hipervinculos sin cajitas rojas

\usepackage[spanish, es-tabla, es-nodecimaldot]{babel} 
% texto automatico en espaniol
% "tabla" en vez de "cuadro"
% no reemplaza puntos decimales por comas

\usepackage{listings}
\usepackage{float}
\lstset{basicstyle=\small\ttfamily,columns=fullflexible}



\def\BibTeX{{\rm B\kern-.05em{\sc i\kern-.025em b}\kern-.08em
    T\kern-.1667em\lower.7ex\hbox{E}\kern-.125emX}}
\begin{document}

\title{Retoque fotográfico mediante reconstrucciones geométricas}
\author{\IEEEauthorblockN{Ariel Nowik, Joaquin Mestanza, Rocio Parra, Martina Máspero, Marcelo Regueira}
\IEEEauthorblockA{22.05 - Análisis de Señales y Sistemas Digitales - Grupo 1} \\
\textit{ITBA: Instituto tecnológico de Buenos Aires}\\
Ciudad de Buenos Aires, Argentina
}
\maketitle

\begin{abstract}
En este trabajo se estudiaron diversos métodos de retoque de imagenes para eliminar elementos no deseados presentes en diversas fuentes. Finalmente se procedió a realizar una implementación en función de las tecnicas analizadas seguida de un análisis de sus ventajas y desventajas.
\end{abstract}

\section{Introducción}
El problema elemental a resolver consiste en la eliminación de un objeto no deseado en una imagen.
Naturalmente no es posible ``adivinar'' lo que se encuentre por detrás, ya que requiere información adicional, la cual en principio no es accesible, solo se dispone de la imagen. Por lo tanto la idea es, de algun modo asimilar la zona de la imagen a reemplazar con el resto de la misma. En lo que continua de este trabajo describiremos con un mayor detalle diversos métodos para llevar a cabo este proceso.

\subfile{assd_tpf_lineasinvest.tex}

\section{Desarrollo de difusión}

\subfile{assd_tpf_exemplar_based.tex}


\subsection{etapa 2}

\subsection{etapa 3}

\section{Primera implementación}

\subsection{Resultados - analisís de efectividad}

\section{Mejora A - etapa tal}

\subsection{Resultados - analisís de efectividad}

\section{Mejora B - etapa tal}

\subsection{Resultados - analisís de efectividad}

\section{Desarrollo en distintas infraestructuras}

\subsection{Implementación en pc}

\subsection{Implementación app de Android}

\section{Conclusiones}

\section{Objetivos futuros}


\begin{thebibliography}{00}
\bibitem{b1} A. Criminisi, P. Perez, and K. Toyama, “Region filling and object ´
removal by exemplar-based image inpainting,” IEEE T. Image Process.,
vol. 13, no. 9, pp. 1200–1212, Sep. 2004.

\bibitem{b2} Pierre Buyssens, Maxime Daisy, David Tschumperlé, Olivier Lézoray. Exemplar-based Inpainting:
Technical Review and new Heuristics for better Geometric Reconstructions. IEEE Transactions on
Image Processing, Institute of Electrical and Electronics Engineers, 2015, 24 (6), pp.1809 - 1824.
ff10.1109/TIP.2015.2411437ff. ffhal-01147620f

\bibitem{b3} Guillemot, Christine \& Le Meur, Olivier. (2014). Image Inpainting : Overview and Recent Advances. Signal Processing Magazine, IEEE. 31. 127-144. 10.1109/MSP.2013.2273004. 

\bibitem{patchmatch} C. Barnes, E. Shechtman, A. Finkelstein, and D. B. Goldman, ``Patchmatch: randomized correspondence algorithm for structural image editing,'' ACM Trans. Graph., vol. 28, no. 3, pp. 24:1–24:11, July 2009.

\end{thebibliography}
\end{document}
